\documentclass[12pt, a4paper]{article} % report, article, book, letter, beamer
\usepackage{graphicx}
\graphicspath{images/}
\usepackage[margin=1in]{geometry}
\usepackage{tabularx}
\usepackage{lscape}
\usepackage{float}
\usepackage{tikz}
\usepackage[utf8]{inputenc}
\usepackage[IL2]{fontenc}
\usepackage[czech]{babel}
\usepackage{pdfpages}
\usepackage{listings}
\usepackage{hyperref}
\usepackage[ddmmyyyy]{datetime}
\renewcommand{\dateseparator}{. }

\usepackage{enumitem}

\newlist{notes}{enumerate}{1}
\setlist[notes]{label=Poznámka: ,leftmargin=*}

\begin{document}
	\begin{titlepage}
		\begin{figure}[H]
			\includegraphics[width=6cm]{images/FAV_cmyk.eps}
		\end{figure}
		\vspace{3cm}
		\begin{center}
			\huge{\textbf{Síťová hra - Lodě}} \\
			\Large KIV/UPS - Semestrální práce
		\end{center}
		\vfill
		\begin{table}[H]
			\begin{tabular}{l l}
				student: & Petr Vondrovic\\
				studijní číslo: & A20B0273P\\
				email: & vondrp@students.zcu.cz\\
				datum: & \today
			\end{tabular}
		\end{table}
	\end{titlepage}
	\pagebreak
	\setcounter{page}{2}
	\section{Pravidla hry}
	\par Lodě, popřípadě Námořní bitva je hra pro dva hráče, ve které je cílem potom všechny lodě soupeře. Oba hráči mají vlastní herní pole o velikosti 10x10 na němž jsou rozmístěny jejich lodě různých velikostí.
	
	\par Lodě mohou být rozmístění pouze vodorovně nebo horizontálně. Je tedy vyloučeno umístění úhlopříčné nebo jinak šikmé. Lodě se navzájem nesmí překrývat a vzájemně se dotýkat a to ani rohem. Během hry se pozice lodí nemění.
	
	
	\par Při hře se účastnící střídají ve  „střelbě“ na jednotlivá pole protivníka, který oznámí zda došlo nebo nedošlo k zásahu, popřípadě zda byla loď zásahem potopena. Byla-li loď potopena je její okolí označeno jako minutí.
	
	\par Hra končí, když jeden z hráčů přijde o celou flotilu.
		
	\section{Implementace programu}
	\par Kromě implementace hry podle specifikovaných pravidel obsahuje hra systém místností. Místnost vytvoří jeden z hráčů a jakmile se k němu připojí protivník hra započne. Každá místnosti má své identifikační číslo a hráči mohou získat jejich seznam, který obsahuje jména hráčů v místnosti.
		
	\par Dojde-li ke ztrátě spojení nebo odpojení se, je stav hry a klienta stále k dispozici a uživatel se může připojit zpět.

	\section{Komunikační protokol}
	\par Komunikační protokol je nešifrovaný a textový. Je tak zajištěna jeho srozumitelnost pro vývoj, testování a kontrolu. Zprávy jsou dělené na 4 druhy a to žádosti, potvrzovací, datové a chybové. Žádosti odesílá klient a přijíma server, zatímco u ostatních typů zpráv je to naopak. Zprávy protokolu využívají k dělení jednotlivých parametrů zpráv středník (;). Každá zpráva je na samostatném řádku, jsou tedy zakončené znakem ($\backslash$n);
	
	\subsection{Stav uživatele}
	\par Klient se může nacházet v jednom z několika stavů. Na základě stavu se odvozeno, které akce mohou být prováděny.
	
	\par Seznam stavů:
	\begin{itemize}
		\item 0 - Nepřihlášený
		\item 1 - V lobby
		\item 2 - V místnosti
		\item 3 - Ve hře - připravuje hrací plochu
		\item 4 - Ve hře - čeká
		\item 5 - Ve hře - na tahu		
	\end{itemize}
	\par Na serveru se nachází ještě dva bonusové stavy - CONNECTED a DISCONNECTED, které sledují zda je klient připojený. Stavy přechází pouze mezi sebou

	\subsection{Stavový diagram}
	\label{ssec:states}
	\par Všechny stavy se mohou dostat do nepřipojeného/nepřihlášeného stavu a po připojení zpět se vrátit.
	\begin{figure}[H]
		\centering
		\begin{tikzpicture}[node distance={25mm}, thick, main/.style = {draw, circle}]
			\node[main] (0) {$0$}; 
			\node[main] (1) [above right of=0] {$1$};
			\node[main] (2) [below right of=0] {$2$}; 
			\node[main] (3) [below of=2] {$3$};
			\node[main] (4) [right of=3] {$4$};
			\node[main] (5) [right of=1] {$5$}; 

			
			\draw[->] (0) to node[midway, above, sloped, pos=0.5] {LOGIN} (1);
			\draw[->] (1) to node[midway, above, sloped, pos=0.5] {JOIN} (2);
			\draw[->] (2) to node[midway, below, sloped, pos=0.5] {PREPARE} (3);
			
			\draw[->] (3) to node[midway, above, sloped, pos=0.5] {START} (4);
			\draw[->] (3) to node[midway, below, sloped, pos=0.7] {START FIRST} (5);
			
			\draw[->] (5) to node[midway, above, sloped, pos=0.5] {END} (1);
			\draw[->] (4) to node[midway, above, sloped, pos=0.8] {END} (1);
			
			\draw[->] (5) to node[midway, above, sloped, pos=0.5] {TAH} (4);
			\draw[->] (4) to [out=0,in=0,looseness=1.0] node[midway, above, sloped, pos=0.5] {TAH} (5);
		\end{tikzpicture}
		\caption{Stavový diagram} 
	\end{figure}
	\subsection{Chybové kódy}
	\label{ssec:error_codes}
	\begin{enumerate}
		\item Interní chyba
		\item Špatný formát zprávy
		\item Špatný formát uživatelského jména
		\item Uživatelské jméno je zabrané
		\item Neplatný stav uživatele
		\item Místnost je plná
		\item Místnost je nedostupná (nebyla nalezena)
		\item Mimo rozsah herního plánu
		\item Už zasažené pole
		\item Neplatné umístění lodě
		\item Neplatný identifikátor lodě
		\item Vyčerpaná kapacita serveru
		\item Neplatná délka lodí
	\end{enumerate}
	\subsection{Skupiny zpráv}
	\par U všech zpráv s vyjímkou navázání spojení může server potvrdit správnost příkazu nebo ohlásit chybu.
	\subsubsection{Navázání spojení}
	\par \uv{connected} zpráva serveru zaslaná při úspěšném navázání spojení klienta se serverem.\\
	\textbf{Zpráva serveru:}\\
	\texttt{connected}
	\subsubsection{Připojení}
	\par Po navázání spojení se klient může přihlásit příkazem \uv{login\_req;(name)}. Na základě stavu se hráč připojí buď do lobi nebo hry/místnosti z níž odešel (při spadnutí klienta). Uživatelské jméno nesmí být delší než 19 znaků a může obsahovat pouze písmena malé, velké abecedy a číslice.
	\par Pokud připojení selže je serverem ukončeno spojení s klientem.\\
	\textbf{Požadavek klienta:}\\
	\texttt{login\_req;(name)}\\
	\textbf{Možné odpovědi serveru:}\\
	\texttt{login\_ok;(name);(state)}\\
	\texttt{login\_err;(error\_code)}\\\\
	\texttt{(name)} - jméno, kterým se klient přihlašuje\\
	\texttt{(error\_code)} - identifikační číslo chyby (viz. \nameref{ssec:error_codes})
	\subsubsection{Vytvoření místnosti}
	\par Žádost o vytvoření místnosti se provádí pomocí zprávy \uv{room\_create\_req}.\\
	\textbf{Požadavek klienta:}\\
	\texttt{room\_create\_req}\\
	\textbf{Možné odpovědi serveru:}\\
	\texttt{room\_create\_ok}\\
	\texttt{room\_create\_err;(error\_code)}\\\\
	\texttt{(error\_code)} - identifikační číslo chyby (viz. \nameref{ssec:error_codes})
	\subsubsection{Seznam místností}
	\par Pro připojení do místnosti je třeba získat její identifikační číslo na nějž se klient ptá příkazem \uv{room\_list\_req} a je mu odpovězeno zprávou \uv{room\_list\_data;(id1);(name1);…} za níž se nachází dvojice identifikátoru místnosti a hráče, který v ní čeká.\\	
	\textbf{Požadavek klienta:}\\
	\texttt{room\_list\_req}\\
	\textbf{Možné odpovědi serveru:}\\
	\texttt{room\_list\_data;(id1);(name1);(id2);(name2)...}\\
	\texttt{room\_list\_err|(error\_code)}\\\\
	\texttt{(idX)} - identifikátor místnosti - nezáporné celé číslo\\
	\texttt{(nameX)} - jméno uživatele, který je aktuálně v místnosti\\
	\texttt{(error\_code)} - identifikační číslo chyby (viz. \nameref{ssec:error_codes})
	\subsubsection{Připojení do místnosti}
	\par S identifikátorem místnosti se klient může přihlásit příkazem \uv{room\_join\_req;(room\_id)}. Povedlo-li se připojení dostanu klient zprávu s jménem protivníka \uv{room\_join\_ok;(opp\_name)}.  Hráč čekající v místnosti dostane zprávu s jménem připojeného uživatele \uv{room\_join\_opp;(name)}.\\	
	\textbf{Požadavek klienta:}\\
	\texttt{room\_join\_req;(id)}\\
	\textbf{Možné odpovědi serveru:}\\
	\texttt{room\_join\_ok;(opp\_name)}\\
	\texttt{room\_join\_err;(error\_code)}\\
	\textbf{Zprávy protihráči:}\\
	\texttt{room\_join\_opp;(name)}\\\\
	\texttt{(id)} - identifikátor místnosti - nezáporné celé číslo\\
	\texttt{(opp\_name)} - jméno protihráče, který je v místnosti\\
	\texttt{(name)} - jméno připojujícího protihráče\\
	\texttt{(error\_code)} - identifikační číslo chyby (viz. \nameref{ssec:error_codes})
	\subsubsection{Odejít z místnosti}
	\par Ze hry/místnosti se odchází příkazem \uv{room\_leave\_req}. Oponent, dostane zprávu o odpojení protihráče \uv{room\_leave\_opp}. Odešel-li hráč během hry, hra skončí.\\
	\textbf{Požadavek klienta:}\\
	\texttt{room\_leave\_req}\\
	\textbf{Možné odpovědi serveru:}\\
	\texttt{room\_leave\_ok}\\
	\texttt{room\_leave\_err;(error\_code)}\\
	\textbf{Zprávy protihráči:}\\
	\texttt{room\_leave\_opp;(name)}\\\\
	\texttt{(name)} - jméno odpojeného protihráče\\
	\texttt{(error\_code)} - identifikační číslo chyby (viz. \nameref{ssec:error_codes})
	\subsubsection{Vstup do hry}
	\par Po té co se do místnosti připojí oba hráči dostanou od serveru zprávu \uv{game\_conn}. Tato zpráva značí, že oba hráči nyní přejdou do fáze přípravy hry.\\
	\textbf{Zpráva serveru:}\\
	\texttt{game\_conn}
	\subsubsection{Herní plocha připravena}
	\label{ssec:game_ready}
	\par Pro vstupu do hry hráči rozmístí své lodě a pošlou svou hrací plochu. Plocha je ve formě řetězce tvořen dvěma typy znaků – E (prázdno) a číslice v rozsahu 0 až 6 jakožto symboly jednotlivých lodí. Příkaz je \uv{game\_prepared;(EEE0000EEE…)}.
	\par Lodě mají pevně stanovenou délku - loď 0 je dlouha 4 políčka, 1 a 2 mají délku 3, 3, 4 a 5 délka 2 a lod s indexem 6 má délku 1.
	\par Po připravení hracích polí obou hráčů dostane jeden z nich zprávu \uv{game\_play} značící začátek hry.\\
	\textbf{Požadavek klienta:}\\
	\texttt{game\_prepared;(EEE0000EEE…)}\\
	\textbf{Možné odpovědi serveru:}\\
	\texttt{game\_prepared\_ok}\\
	\texttt{game\_prepared\_err;(error\_code)}\\
	\texttt{game\_play}\\\\
	\texttt{(name)} - jméno odpojeného protihráče\\
	\texttt{(error\_code)} - identifikační číslo chyby (viz. \nameref{ssec:error_codes})
		
	\begin{notes}
		\item Informace o lodí jsou zaslány v podobě jejich identifikátorů, protože ten pomáhá serveru a klientovi označit okolí lodě při jejím zničení.
	\end{notes}	
	\subsubsection{Střelba}
	\par Příkaz vystřelit na danou pozici \uv{game\_fire\_req;(x);(y)}. Bylo-li vystřeleno na pozici v platném rozsahu, která ještě zasažena nebyla, dostane klient od serveru zprávu\\ \uv{game\_fire\_ok;(x),(y),(status);(ship\_destroyed)} informující o stavu zasažené pozice a zničení lodě. Při úspěchu dostane protihráč informaci o zasaženém poli\\ \uv{game\_opp\_fire;(x);(y);(status);(ship\_destroyed)}.\\
	\textbf{Požadavek klienta:}\\
	\texttt{game\_fire\_req;(x);(y)}\\
	\textbf{Možné odpovědi serveru:}\\
	\texttt{game\_fire\_ok;(x),(y),(status);(ship\_destroyed)}\\
	\texttt{game\_fire\_err;(error\_code)}\\
	\textbf{Zprávy protihráči:}\\
	\texttt{game\_opp\_fire;(x);(y);(status);(ship\_destroyed)}\\\\
	\texttt{(x);(y)} - x-ová a  y-ová souřadnice \\
	\texttt{(status)} - stav zasaženého pole - H (loď zasažena), M (minul)\\
	\texttt{(ship\_destroyed) - 0 - loď žije, 1 - loď zničena}\\
	\texttt{(error\_code)} - identifikační číslo chyby (viz. \nameref{ssec:error_codes})
	\subsubsection{Informace o hře}
	\par Pro znovu připojení nebo synchronizace může klient vyslat žádost o data.\\
	\textbf{Požadavek klienta:}\\
	\texttt{game\_info\_req}\\
	\textbf{Možné odpovědi serveru:}\\
	\texttt{game\_info\_data;(client\_status);(client\_board);(opp\_name);(opp\_board)}\\
	\texttt{game\_info\_err|(error\_code)}\\\\
	\texttt{(client\_status)} - status klienta (viz. \nameref{ssec:states})\\
	\texttt{(opp\_name)} - jméno protihráče\\
	\texttt{(client\_board)} - řetězcová reprezentace klientova hracího plánu. Obsahuje symboly \texttt{E} - prázdné pole, \texttt{H} - zasažená loď, \texttt{M} - zasažené prázdné pole a číslice od \texttt{0} do \texttt{6} symbolující jednotlivé lodě.\\\\\\
	\texttt{(opp\_board)} - řetězcová reprezentace protivního hracího plánu. Obsahuje symboly \texttt{E} - prázdné pole, \texttt{H} - zasažená loď a \texttt{M} - zasažené prázdné pole.\\
	\texttt{(error\_code)} - identifikační číslo chyby (viz. \nameref{ssec:error_codes})
	
	\subsubsection{Ukončení hry}
	\par Zprávu zasílá server klientům s parametrem jména vítěze.\\
	\textbf{Zpráva serveru:}\\
	\texttt{game\_end;(winner\_name)}\\
	\texttt{winner\_name} - jméno vítěze
	\subsubsection{Odpojení}
	\par Příkaz s žádostí o odpojení od serveru je \uv{logout\_req}. Server odpoví buď s \uv{logout\_ok}, případně s chybou \uv{logout\_err;(error\_code)}.\\
	\textbf{Požadavek klienta:}\\
	\texttt{logout\_req}\\
	\textbf{Možné odpovědi serveru:}\\
	\texttt{logout\_ok}\\
	\texttt{logout\_err;(error\_code)}\\
	\textbf{Zprávy protihráči:}\\
	\texttt{room\_leave\_opp;(name)}\\\\
	\texttt{(name)} - jméno odpojeného spoluhráče\\
	\texttt{(error\_code)} - identifikační číslo chyby (viz. \nameref{ssec:error_codes})
	
	\section{Implementace serveru}
	\par Server je implementovaný v programovacím jazyce C. Pro realizace paralelního serveru je použita funkce \texttt{select}, zbytek serveru je realizován v jednom vlákně. Server je tvořen strukturama pro jednotlivá připojení a data ukládá do datových struktur. Například jednolitvé místnosti/hry mají přiřazené identifikátor, kterým jsou rychle získatelné z array listu. Uživatelé jsou ukládány v hashmapě, kde je lze rychle vyhledat pomocí jejich jména.
	
	\par Informace o činnostech serveru jsou zasílána na výstup a do souboru \texttt{trace.log} se ukládají trasovací informace. Po ukončení serveru za pomocí signalu (např. CTRL + C) je vytvořen soubor \texttt{stats.txt}, kde jsou uloženy statistické údaje o činnosti serveru.

	\subsection{Popis složek a jednotlivých souborů}
	\subsubsection{main.c/h}
	\par Soubor funkcí se stará o vstup do programu inicializaci základních struktur, a procesování jednotlivých zpráv.
	\subsubsection{Složka structures}
	\par Ve složce se nachází výše zmíněné datové struktury array listu a hashmapy.
	\subsubsection{Složka game}
	\par Ve složce game se nachází soubory \texttt{ship.c/h} a \texttt{shipsGame.c/h}. Ty slouží jakožto modely lodí a hry/místnosti a funkce sloužící k jejich nastavení.
	\subsubsection{Složka communication}
	\par Složka obsahuje soubory starající se o obsluhu připojení.\\
	\par \texttt{server.c/h} se stará o udržování informací napříč dalšími funkcemi,  obsluhuje nová připojení a odpojuje ty ukončující.
	\par Soubor \texttt{client.c/h} obstarává nastavení klienta.
	\par \texttt{commands.c/h} obsahuje seznam všech očekávajících příchozích zpráv a jejich obsluhu.
	\par V souboru \texttt{errors.h} se nachází přehled definovaných chybových stavů.	
	
	\section{Implementace klienta}
	\par Klient je implementovaný v jazyce Java ve verzi 11, s použitím grafické knihovny JavaFX.
	\subsection{Rozdělení}
	\par Aplikace je implementovaná pomocí MVC modelu (model, view, controller). 
	View vrstva se nachází ve složce resources a skládá se z fxml souborů jednotlivých oken aplikace. Ve vztahu 1 ku 1  se ve složce controller se nachází kontrolery oken, které se starají o propojení pohledů s modely.\\
	
	\subsection{Nejdůležitější části modelu připojení}	
	\subsubsection{Reciever}
	\par Třída reprezentuje samostatné vlákno, uzavřené v nekonečné smyčce. Vlákno přijímá zprávy ze serveru. A posílá je na zpracování třídě \texttt{MessageHandler}.
	
	\subsubsection{MessageHandler}
	\par Instance třídy obsluhuje jednotlivé možné zprávy ze serveru a vykonává jejich obslužné akce. Pokud počet po sobě jdoucích zpráv, které neodpovídají žádné zprávě, překročí nastavenou mez, klient se od serveru odpojí.
	
	
	\subsection{Nejdůležitější části modelud hry}
	\subsubsection{GameModel}
	\par Obsahuje data herních ploch, informaci o vítězi a současný stav hry. Nachází se zde metody sloužící k inicializaci herních plánů, rozmístění lodí a obstarání  zasažení pole herního plánu.
	
	\subsubsection{GameBoard}
	\par Grafická reprezentace herního plánu zobrazující data z \texttt{GameModel}.
	\subsection{Hlavní}
	\subsubsection{App}
	\par Třída reprezentující grafickou aplikaci. Jedná se o třídu s návrhovým vzorem jedináček, jejíž  instance shlukuje odkazy na jednotlivé moduly aplikace. Třída je volána spouštěcí třídou \texttt{Main}.
	
	\section{Uživatelská dokumentace}
	\subsection{Překlad serveru}
	\par Pro překlad serveru je potřeba nástroj \texttt{cmake}, \texttt{make} a překladač \texttt{gcc}. Nejprve vygenerujeme soubor makefile zadáním příkazu \texttt{cmake -B ./build/}. Přejdeme do adresáře \texttt{build} příkazem \texttt{cd build} a v něm spustíme překlad programu příkazem \texttt{make}. Po dokončení operace by měl být vytvořen v adresáři \texttt{build} soubor \texttt{server}. 
	\subsection{Spuštění serveru}
	\par Pro překladu programu se vyskytujeme v adresáři \texttt{/server/}. Pro build aplikace je používán nástroj CMake, jenž vytvoří příslušný Makefile.\\
	\par Použijeme tedy příkaz: \texttt{cmake -B ./build/}\\
	\par Přemístíme se do adresáře \texttt{build/} (\texttt{cd build}) a provedeme sestavení spustitelného souboru příkazem \texttt{make}, který vytvoří spustitelný soubor \texttt{server}.
	\par Pro spuštění serveru potřebujeme znát číslo portu služby, například \texttt{9123}. Server s tímto portem spustíme příkazem:
	\texttt{./server -p 9123} nebo \texttt{./server --port 9123}\\
	\par V případě kdy uvedeme neplatnou hodnotu portu - tedy nečíselnou hodnotu, číslo menší než \texttt{1} nebo větší než texttt{65535} dojde k nastavení defaultní hodnoty portu \texttt{9123}\\
	\par Program můžeme spusti i s několika nepovinnými parametry:
	\begin{itemize}
		\item Maximální počet místností: \texttt{-r 15} nebo \texttt{--rooms 15}  (defaultní hodnota 10)
		\item Maximální počet uživatelů: \texttt{-pl 30} nebo \texttt{--players 30} (defaultní hodnota 20)
		\item IP adresa: \texttt{-ip 127.0.5.5} - bez defaultní hodnoty - server sdělí zda povedlo s danou IP adresou spustit
	\end{itemize}
	\par Výsledné hodnoty parametrů se vypíšou do konzole.
	

	\subsection{Překlad a spuštění klienta}
	\par Klient se spouští pomocí služby \texttt{gradle}. Program se přeloží, sestaví a spustí příkazem \texttt{./gradlew run} použitým ve složce \texttt{client}.
	\par Můžeme si nechat vygenerovat spustitelný JAR příkazem \texttt{./gradlew jar}. Vygenerovaný JAR soubor se uloží do \texttt{build/libs/} pod názvem \texttt{client.jar}. Máme-li správně nainstalované potřebné knihovny a nastavené cesty, můžeme tento soubor spustit příkazem \texttt{java -jar client.jar}
	\section{Závěr}
	\par Server a klient spolu komunikují na popsaném protokolu a je přes něj realizovaná síťová hra. Program ošetřuje nevalidní zprávy, reaguje na výpadky a běží stabilně. Požadavky na práci jsou splněny.
	
\end{document}